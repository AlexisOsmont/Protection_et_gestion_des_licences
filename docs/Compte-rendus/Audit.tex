\documentclass{article}
\usepackage[utf8]{inputenc}
\usepackage{hyperref}
\usepackage{fontawesome}
\usepackage[margin=3.8cm]{geometry}
\usepackage[french]{babel}
\usepackage[backend=bibtex8]{biblatex}

\title{
    \Huge
    Compte-rendu d'audit\\
    de gestion de projet\\
}
\date{\huge 18 Mars 2022}
\author{\huge Alexis Osmont\\}

\newcommand{\arrow}{$\rightarrow$ }

\begin{document}

\maketitle
\vspace{5cm}
    Présent à la réunion :
    \begin{description}
        \item Alexis Osmont
        \item Kaci Hammoudi (distanciel)
        \item Louka Boivin
        \item Noé Dallet
        \item Sami Babigeon
    \end{description}
\newpage

\section{Introduction}   

La réunion d'audit avait pour but de réaliser une revu du projet dans son ensemble pour nous permettre 
d'améliorer notre gestion de projet ce qui, de facons générales permet de mener le projet à bien. La réunion
a donc traité les sujets suivant : 
Introduction et rappel du contexte de l’audit,
Avancement du projet,
Échanges autour de votre organisation et 
Définition d’actions associées.\\
\newline
A fin de réaliser un audit le chef de projet a donc réalisé une courte presentation pour aborder les sujet ci-dessus
Cette presentation a suivit le plan suivant :
\begin{itemize}
    \item Situation actuelle du projet
    \item Revue du planning global (Product Roadmap) et des risques
    \item Tenue des engagements (date et livrables, recette client)
    \item Charges prévues, consommées et reste à faire par itération (burndown chart)
    \item Retour sur l’organisation de des premières itérations de srpint (Sprint Retrospective)
    \item Nouveaux besoins identifiés et plan d’action associé\newline
\end{itemize}

De plus nous avons présenté la derniere version des documents fournis pour la revue de lancement de projet et cours 
de cette même presentation.

\vspace{0.5cm}

\section{Retour de l'audit}
\subsection{Points soulevés}
\begin{itemize}
    \item 1 sprint réalisé en 4 semaines plutôt que 3 2 ème sprint en cours suite à l’absence du client
    \item Dernière réunion avec le client le 7 mars
    \item Délais tenus sur les sprints (cf. Gantt) et tâches également
    \item Manque de communication vu dans la rétrospective (MetroRetro)
    \item Les livrables conviennent au client
    \item Durée des sprints revus pour basculer à 2 semaines
\end{itemize}

\subsection{Elements à risque}
\begin{itemize}
    \item Validation des livrables des itérations ? > Non formalisée
    \item Présentation des livrables du projet ? Délais de validation ? > Informelle
    \item Validation du client sur les deux sprints ? > Informelle
    \item Validation du client sur le périmètre du sprint en cours ? CR de réunion ? > OK mais ne couvrent pas les décisions
    \item Suivi des tâches des charges passées
\end{itemize}
\newpage
\section{Amélioration à apporter}
\subsection{Actions requises}
\begin{itemize}
    \item Réaliser un rapport de tests pour chacune des itérations
    \item Formaliser les livraisons et demander au client de valider sur la base de votre rapport de tests couverts par l’équipe
    \item Réaliser un graphique d’avancement de la charge restante (burndown chart) à la journée par sprint
    \item Gestion des risques « Absence du client » décrivant la solution proposée, le plan d’action et la solution technique
    \item Définir un outil de suivi des tâches > Saisir les charges prévues/passées/restantes
    \item Prochain CR de rétrospective de sprint (ex. Export de MetroRetro)
\end{itemize}

\newpage


\end{document}
