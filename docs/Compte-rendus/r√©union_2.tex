\documentclass{article}
\usepackage[utf8]{inputenc}
\usepackage{hyperref}
\usepackage{fontawesome}
\usepackage[margin=3.8cm]{geometry}
\usepackage[french]{babel}
\usepackage[backend=bibtex8]{biblatex}

\title{
    \Huge
    Compte-rendu\\
    Réunion numéro 2\\
}
\date{\huge 23 Novembre 2021}
\author{\huge Alexis Osmont\\}

\newcommand{\arrow}{$\rightarrow$ }

\begin{document}

\maketitle
\vspace{5cm}
    Présent à la réunion :
    \begin{description}
        \item Alexis Osmont
        \item Kaci Hammoudi
        \item Louka  Boivin
        \item Sami Babigeon
    \end{description}
\newpage

\section{Sujet de réunion}                                                                                 

Le but de cette réunion était de présenter puis d'avoir un retour sur le document de spécification technique des besoins (STB) pour attester de la bonne direction du projet.\\

Nous avons aussi abordé le sujet du document d'architecture technique (DAT), ainsi qu'une présentation de schémas et des discussions quant à la gestion de projet.
\vspace{0.5cm}

\section{Retour sur compte rendu}
\subsection{STB}

\begin{itemize}
    \item Attention aux standards de rédaction de documents techniques.
    \item Révision de l'orthographe et des formulations.
    \item Réorganisation des sections (section 5).
    \item Epuration de certaines sections du document.
    \item Ajout de l'exigence "code maintenable et réutilisable".
    \item Revoir les règles de gestion.
\end{itemize}

\subsection{DAT}
\begin{itemize}
    \item Modification du schéma système.
    \item Attention aux flèches et leurs directions.
    \item Définition d'une priorité : Greffe de code

\end{itemize}
\section{Discussions d'aspects technique}

Certains points techniques ont été abordé durant cette réunion, tel que la greffe de code et sa priorité, l'importance de
multiples vérifications de licence dans le code et la prise en compte de certains moyens de contournement de notre projet.\\

Un element essentiel est la maintenabilité du projet et de ses extensions afin que le code soit réutilisable au cours de 
futur projet ou puisse subir une maintenance en cas de panne.\\

L'obfuscation est un des autres élements abordé comme une contre-mesure face au reverse engineering, le fait d'encoder les chaines constantes,
ou encore de modifier les noms de variables et les commentaires sont importants.\\

Plusieurs cas de génération de licences ou de modification de code ont été proposé comme cas particuliers de tentatives de casse. Par exemple 
la modification de la date de la machine pour obtenir une licence infinie, le changement de serveur pour obtenir des licences, 
le faux formatage de machine pour obtenir le logiciel gratuitement sur une autre machine qui pourrait être évité avec des fonctions 
de suppression de logiciel ou d'invalidation de licence. 

\newpage


\end{document}
