\chapter{Terminologies}

\begin{itemize}
	\item Le client est le commanditaire du projet.
	\item Un utilisateur est une personne souhaitant utiliser un logiciel du client. 
	\item Une licence est un droit accordé pour une machine et un utilisateur d'utiliser un logiciel donné.
	\item Craquer un logiciel est le fait de pouvoir l'utiliser sans avoir payé pour son utilisation. 
	Soit en modifiant le code compilé, soit en utilisant une autre méthode. 
\end{itemize}

\chapter{Présentation}

\chapter{Problèmes de date}

Aujourd'hui au cours de notre projet nous nous sommes rendu compte que le bypass de 
date de licence pouvait se faire assez facilement, par exemple en changeant la date de la machine 
afin de revenir en arriere et profiter d'une licence infini dans le cas ou l'utilisateur
ne possede pas de connexion internet (volontairement ou non).\newline

Nous avons donc etudié se domaine un peu plus en profondeur ce qui nous a donné cette
preuve de concept qui aborde les failles, les solutions et la manière de les mettre en place.\newline

\section{Failles et solutions}

	La premiere faille abordée plus haut, est celle du changement simple de date pour revenir en arriere
	pour profiter d'un temps d'utilisation ilimité.\newline

	Pour corriger cette faille potentielle il nous suiffit de stocker de manière chiffré la date de création de la licence
	pour qu'a chaque utilisation du logiciel celui-ci vérifie que la date actuelle est bien superieur ou egale à celle
	stockée.\newline

	La seconde faille est celle de la date bloquée à un jour fixe et a une heure fixe. \newline

  La solution de cette faille est pratiquement identique à celle de la premiere. Cependant pour cette fois il faut incrementer la date
  a chaque utilisation du logiciel qui pourrait compter le temps par lui même. \newline

  La troixieme faille qui decoule des 2 premiere nous montre un cas partculier. En effet dans le cas ou la date est incrementée, cela donne la prossibilité à l'utilisateur
	d'utiliser la licence 24h sur 24. Plus précisement, dans le cas ou l'utilisateur utilise le logiciel et une fois arreté decide de remettre la derniere date a laquelle 
	celui-ci l'avait utilisé. Cela donnerait une utilisation du logiciel  non plus en jours de licence mais en temps d'utilisation.\newline

	L'unique solution trouvée pour cette faille est le fait de chercher dans les log ou dans un autre fichier une preuve de date modifiée.\newline

\section{Solution globale}

La solution globale trouvée consiste à combiner toutes les solutions abordées précedement en même temps pour rendre la tache plus compliqué aux fraudeurs.\newline

\begin{itemize}
	\item Création d'un demon
	\item Création d'un fichier signé/chiffré pour stoker et incrementer une date  
	\item Vérifier les logs pour identifier des changements de date
	\item Constament chercher internet afin de vérifier la date actuelle\newline
\end{itemize}

Malgrés ça une erreur perciste, le cas de l'utilisation 24H/24 du logiciel dans le cas ou aucune preuve n'a été trouvée dans les logs.

C'est pourquoi nous vous proposons cette solution :

\chapter{Injection de code}

\section{Solutions}

\label{chapter:bilan}

