\chapter{Objet}

Ce document a pour but de présenter l'architecture technique de ce projet. Le but de ce projet est de développer un outil de protection et de gestion des licences.
Cette outil comprendra plusieurs composants :
\begin{itemize}
    \item Une plateforme web d'enregistrement et de gestion des licences.
    \item Un logiciel d'activation de licence.
    \item Un outil permettant d'ajouter la vérification de la licence à une application.
\end{itemize}
Ce dernier pourra se présenter de 2 manières différentes.
Dans un premier temps il s'agira d'une bibliothèque de fonctions permettant à un développeur d'intégrer
la vérification de la licence à son code.
Dans un second temps la vérification sera gréffée directement sur un éxécutable.\newline

Ce document présente et détaille donc l'architecture du projet et explique le fonctionnement des compososants et leur interactions.
Également, il a pour but de mettre en avant les solutions apportées aux besoins du client et les moyens de répondre aux exigences
fonctionnelles et techniques.

\chapter{Documents de référence}

Documents de spécification :

\begin{itemize}
    \item Spécification technique des besoins
    \item Fiches techniques :
	\begin{itemize}
	    \item Injection de code dans un PE
	    \item Génération de licence
	    \item Signature
	    \item Obfuscation de code
	    \item Sécurisation des bases de données
	\end{itemize}
\end{itemize}

\chapter{Configuration requise}

Les éléments matériels nécessaire au développement de cet outil sont :
\begin{itemize}
    \item Un serveur sur lequel nous devrons installer le necéssaire à la mise en place d'un
        serveur Web (Apache, Tomcat, PHP, MySQL).
    \item Un ordinateur sous Windows pour développer les interfaces et effectuer les tests.
\end{itemize}

De plus nous aurons besoin de différents logiciels pour effectuer des tests (protéger les
logiciels) lorsque le développement de l'outil sera suffisamment avancé.

\chapter{Architecture statique}
Afin de réaliser ce projet nous avons choisi de mettre en place l'infrastructure suivante:

\begin{description}
	\item[\textbf{Un serveur web}:]
		C'est ce serveur qui présentera l'interface aux utilisateurs et :
		\begin{itemize}
			\item Permettra à un client de se créer un compte, demander une licence et 
			télécharger l'outil de génération de licence.
			\item Permettra à un administrateur d'accorder et de gérer les licences. 
		\end{itemize}
	\item[\textbf{Un logiciel d'activation}:] 
		Sera executé sur la machine de l'utilisateur
		afin de lui permettre de générer une licence. Pour cela il échangera avec
		le serveur des informations sur le matériel physique de la machine, ce 
		qui permettra à ce dernier de générer le fichier de licence.
	\item[\textbf{Outil de validation}:]
		Outil permettant de vérifier la validité d'une licence, cet outil
		pourra prendre les formes suivantes: 
		\begin{itemize}
		\item Une bibliothèque de fonctions qui permettra à un developpeur d'inclure
		dans son code les fonctions de vérification de licence. 
		\item Un programme permettant de greffer directement les fonctions de vérification
		sur un exécutable. 
		\end{itemize}
\end{description}

L'architecture générale de ce projet peut être représenter via 
le schéma n°\ref{fig:fig1}.
\newpage

\begin{figure}[hp!]
	\centering
	\includegraphics[width=\textwidth]{../png/DAT_general.png}
	\caption{Schéma d'architecture générale}
	\label{fig:fig1}
\end{figure}

\newpage

\section{Constituant «application»}

\begin{figure}[p]
    \vspace{-0.5cm}
	\caption{API rest}
	\definecolor{getColor}{rgb}{0.28, 0.8, 0.56}
\definecolor{postColor}{rgb}{0.38, 0.68, 0.99}

\usemintedstyle{monokai}
\definecolor{bg}{HTML}{282828} % from https://github.com/kevinsawicki/monokai

\begin{tcolorbox}[title=RestFull API]

	\begin{tcbitemize}[%
    raster columns=7, 
    raster rows=2, 
    raster equal height=rows,
    ]

		% Row I
		\tcbitem[myhbox={}{},
			colback=getColor,colupper=white, 
			fontupper=\sffamily\bfseries\normalsize]	GET
		\tcbitem[raster multicolumn=6, myhbox={}{},
						 fontupper=\ttfamily\bfseries\normalsize] /api/v1/Software/getSoftwareList

		% Row II
		\tcbitem[raster multicolumn=5, myhbox={}{}, 
						 frame hidden, interior hidden] \textbf{Parameters:} 

		\tcbitem[raster multicolumn=2, colback=white, 
						 colupper=black, myhbox={}{}] No parameters 

		% Row III
		\tcbitem[raster multicolumn=3, myhbox={}{}, 
						 frame hidden, interior hidden] \textbf{Response:} 

		\tcbitem[raster multicolumn=2, myhbox={}{}, 
						 frame hidden, interior hidden] \emph{content type}

		\tcbitem[raster multicolumn=2, colback=white, 
						 colupper=black, myhbox={}{}] application/json

		% Row IV
		\tcbitem[raster multicolumn=7, myhbox={}{}]
		\begin{tcbitemize}[raster columns=7, 
											 raster rows=2]
				
			\tcbitem[raster multirow=1, myhbox={}{},
							 fontupper=\sffamily\bfseries\normalsize,
							 frame hidden, interior hidden] 200

			\tcbitem[raster multirow=1, raster multicolumn=6, myhbox={}{}, colback=bg] 
					\begin{minted}[bgcolor=bg, breaklines, breakautoindent=true]{json}
[{
	"SoftwareId" : 0,
	"SoftwareName" : "Logiciel de Gestion de Paye",
	"SoftwareDesc" : "Permet de gérer les payes de vos employés"
}]
				\end{minted}
			
			\tcbitem[raster multirow=1, myhbox={}{},
							 fontupper=\sffamily\bfseries\normalsize,
							 frame hidden, interior hidden] 404

			\tcbitem[raster multirow=1, raster multicolumn=6, myhbox={}{}] Not Found

		\end{tcbitemize}

		% Row V			
		\tcbitem[myhbox={}{},
			colback=postColor,colupper=white, 
			fontupper=\sffamily\bfseries\normalsize]	POST
		\tcbitem[raster multicolumn=6, myhbox={}{},
						 fontupper=\ttfamily\bfseries\normalsize] /api/v1/Licence/requestLicence

	% Row III
	\tcbitem[raster multicolumn=3, myhbox={}{}, 
					 frame hidden, interior hidden] \textbf{Parameters:} 

	\tcbitem[raster multicolumn=2, myhbox={}{}, 
					 frame hidden, interior hidden] \emph{content type}

	\tcbitem[raster multicolumn=2, colback=white, 
					 colupper=black, myhbox={}{}] application/json

	\tcbitem[raster multicolumn=7, myhbox={}{}, colback=bg] 
					\begin{minted}[bgcolor=bg, breaklines, breakautoindent=true]{json}
[{
	"UserMail" : "prenom.nom@mail.com",
	"UserPassword" : "123456seven",
	"SoftwareId" : 0,
	"HardwareHash" : "2e41d4a7-8be7-40ef-8b29-e77aadee37cb"
}]
				\end{minted}
		
	% Row II
	\tcbitem[raster multicolumn=7, myhbox={}{}, 
					 frame hidden, interior hidden] \textbf{Response:} 

	% Row IV
	\tcbitem[raster multicolumn=7, myhbox={}{}]
	\begin{tcbitemize}[raster columns=7, 
										 raster rows=2]
			
		\tcbitem[raster multirow=1, myhbox={}{},
						 fontupper=\sffamily\bfseries\normalsize,
						 frame hidden, interior hidden] 200

		\tcbitem[raster multirow=1, raster multicolumn=6, myhbox={}{}] LicenceFile.key 

		\tcbitem[raster multirow=1, myhbox={}{},
						 fontupper=\sffamily\bfseries\normalsize,
						 frame hidden, interior hidden] 404

		\tcbitem[raster multirow=1, raster multicolumn=6, myhbox={}{}] Not Found

		\end{tcbitemize}

	\end{tcbitemize}
\end{tcolorbox}

	\label{fig:fig2}
\end{figure}

\subsection*{Description}

\begin{description}
	\item[\textbf{Rôle}:]
	    Le but du serveur est de fournir l'interface principale aux utilisateurs sous forme d'une interface Web.
	    Elle permettra d'effectuer toutes les opérations nécéssaires à la création de compte et à la gestion des
	    licences, de stocker dans sa base de données les informations nécessaires à la génération de licence
        et enfin de générer les licences.
	%\item[\textbf{Propriétés et attributs de caractérisation}:]
	\item[\textbf{Services offerts}:]
				API Rest \cite{REST} permettant au client d'envoyer les informations
				nécéssaires à la génération de la licence (voir \ref{fig:fig2}).			
	\item[\textbf{Dépendances}:]
				Dépend du module de base de données et du serveur Tomcat.
	\item[\textbf{Langages}:]
                HTML, CSS, Java, SQL, Bash, C
	\item[\textbf{Procédé de développement}:]
				Utilisation du pattern design MVC (Modèle - Vue - Contrôleur) et respect des
                conventions Material Design \cite{Material} (utilisation d'un framework CSS).
	\item[\textbf{Taille et complexité}:]
				Taille importante et complexité élévé.
\end{description}

\subsection*{Justifications techniques}
Utilisation de Java pour la partie back-end. Le modèle peut être écrit en Java, et la partie
contrôleur utilisera les servlets et les JSP. La programmation orienté-objet avec Java permet
d'avoir un code plus compréhensible et plus simple. De plus, comparé à PHP, Java offre une plus
large API. Avec cette technologie, il y a possibilité d'automatiser simplement le déploiement de
la solution via un script d'installation.

\subsection*{Script de génération}
Le serveur utilisera un script permettant de générer une licence pour un 
utilisateur en prenant en paramètre l'identifiant du logiciel pour lequel on doit 
générer la licence, un hashé des informations hardware du client, et les
attributs de la licence (nombre d'essais, période de validité) et renverra
le fichier de licence.

Les fonctions qui devront être implementés pour ce script sont décrites dans l'annexe
(section \ref{api:gen}).
\newpage

\section{Constituant «Base de données»}

\begin{figure}[!h]
	\centering
	\includegraphics[width=\textwidth - 2cm]{../png/SQL_table.png}
	\caption{Diagramme UML des tables de la BDD}
	\label{fig:fig2}
\end{figure}

\subsection*{Description}
\begin{description}
	\item[\textbf{Rôle}:]
		Le but de ce module est de permettre de stocker les informations relatives aux
		clients, administrateurs et aux licences.
%	\item[\textbf{Propriétés et attributs de caractérisation}:]
	\item[\textbf{Services offerts}:]
		Fonctions SQL permettant de créer / modifier / supprimer / consulter les
		informations relatives aux clients, administrateurs et licences.
	\item[\textbf{Dépendances}:]
		Pas de dépendances.		
	\item[\textbf{Languages}:]
		Java, SQL.
	\item[\textbf{Procédé de développement}:]
	    Porter une attention partculière à la sécurisation des données, via
	    du chiffrement et à se proteger des attaques du type «injections SQL». 
	\item[\textbf{Taille et complexité}:]
		Taille faible et complexité faible. 
\end{description}

\subsection*{Justifications techniques}
MySQL est le SGBD le plus simple et le plus utilisé. De plus celui ci est open source et son
installation est rapide.

\subsection*{Modèle Java et DAO}
Nous utiliserons MySQL avec Java (JDBC) pour interagir avec notre base de données
depuis la plateforme Web.

Le modèle Java correspondra à une classe par types d'objets dans la base de données (c'est-à-dire
une par table). Ensuite à chacune de ces classes correspondra son DAO (Data Access Object) qui
effectuera les requêtes avec la base de données et qui renverra les objets correspondant au modèle.

\section{Constituant «Logiciel d'activation»}

\subsection*{Description}
\begin{description}
	\item[\textbf{Rôle}:]
		Récupérer les informations matérielles sur la machine d'un utilisateur, puis 
		se connecter au serveur afin de générer le fichier de licence.
%	\item[\textbf{Propriétés et attributs de caractérisation}:]
	\item[\textbf{Services offerts}:] 
		N'offre aucun service.
	\item[\textbf{Dépendances}:]
		Dépend du module application.
	\item[\textbf{Languages}:]
		C\#.
	\item[\textbf{Procédé de développement}:]
		Respect d'une charte de code. (voir chapitre \ref{chap:Annexe})
	\item[\textbf{Taille et complexité}:]
		Taille faible et complexité faible. 
\end{description}

\subsection*{Justifications techniques}
Ce logiciel sera principalement constitué d'une interface graphique et fera des requêtes HTTPS sur
notre serveur. Il devra être éxécutable sur Windows et donc, le langage C\# est le plus adapté pour
cette tâche car il permet de réaliser facilement des interfaces Windows.
\newpage

\section{Constituant «Système de vérification - Librairie»}
\subsection*{Description}
\begin{description}
	\item[\textbf{Rôle}:]
		Fournir une API pour des programmes en C\# afin de 
		permettre à un developpeur de vérifier une licence dans son code.
        Cependant le code de la bibliothèque ne sera pas en C\# mais en 
        C car celui-ci est plus facile à obfusquer et plus efficace. 
	\item[\textbf{Services offerts}:]
        Ce module propose une API dont les fonctions sont définies en annexe
        (section \ref{api:lib}).
	\item[\textbf{Dépendances}:]
		Dépend d'un fichier de licence.	
	\item[\textbf{Languages}:]
		C, C++, C\#
	\item[\textbf{Procédé de développement}:]
		Respect d'une charte de code. (voir chapitre \ref{chap:Annexe})
	\item[\textbf{Taille et complexité}:]
		Taille importante et complexité relative.
\end{description}

\subsection*{Justifications techniques}
Dans un premier temps, les fonctions de l'API pourront être implémentées en utilisant des bibliothèques systèmes comme par exemple, pour les informations hardware, la classe \verb:System.Management:, ou bien la classe \verb:System.Security.Cryptography: pour le hash avec SHA-256.\newline

Dans un second temps, nous devrons implanter au maximum ces fonctions nous-mêmes et 
les optimisés dans le but de pouvoir les intégrer directement dans un 
exécutable (greffe dans un PE) sans pour autant alourdir ou ralentir l'éxécution de 
l'application. \newline

C'est aussi dans un but pédagogique nous avons choisi d'implémenter nous même le système cryptographique de signature. Le schéma de signature retenu est ElGamal \cite{ElGamal}.
\newpage

\section{Constituant «Système de vérification - Greffe»}
\subsection*{Description}
\begin{description}
	\item[\textbf{Rôle}:]
			Outil permettant d'ajouter les fonctions de vérification directement à un
			éxécutable, afin de décharger le developpeur de la necessité de vérifier 
			la licence dans son code.
%	\item[\textbf{Propriétés et attributs de caractérisation}:]
	\item[\textbf{Services offerts}:]
		N'offre aucun service.
	\item[\textbf{Dépendances}:]
		Dépend de la bibliothèque de vérification.
	\item[\textbf{Languages}:]
		C, C\#, Python, Bash 
	\item[\textbf{Procédé de développement}:]
		Respect d'une charte de code. (voir chapitre \ref{chap:Annexe}) 
	\item[\textbf{Taille et complexité}:]
		Taille faible et complexité élevé.
\end{description}


\chapter{Fonctionnement dynamique}
Cette section décrit le fonctionnement dynamique des composants selon les cas d'utilisations vu
dans la spécification technique des besoins.

\section{Schémas de séquences techniques}

\subsection{CA1 et CA2 : Demande et accord d'une licence}
\begin{figure}[!h]
    \centering
    \includegraphics[width=\textwidth]{../png/SSD-demande-licence.png}
    \caption{Schéma de séquence technique - Demande et accord d'une licence}
\end{figure}
\newpage

\subsection{CA3 : Activation d'une licence}
\begin{figure}[!h]
    \centering
    \includegraphics[width=15cm]{../png/SSD-logiciel-activation.png}
    \caption{Schéma de séquence technique - Activation d'une licence}
\end{figure}

\subsection{CA4 : Démarrage d'un logiciel protégé}
\begin{figure}[!h]
    \centering
    \includegraphics[width=15cm]{../png/SSD-demarrage-logiciel.png}
    \caption{Schéma de séquence technique - Démarrage d'un logiciel protégé}
\end{figure}
\newpage

\subsection{CA5 : Création d'un compte}
\begin{figure}[!h]
    \centering
    \includegraphics[width=15cm]{../png/SSD-creation-compte.png}
    \caption{Schéma de séquence technique - Création d'un compte}
\end{figure}

\subsection{CA6 : Paramétrage d'une licence}
\begin{figure}[!h]
    \centering
    \includegraphics[width=15cm]{../png/SSD-param-licence.png}
    \caption{Schéma de séquence technique - Paramétrage d'une licence}
\end{figure}
\newpage

\chapter{Annexe}
\label{chap:Annexe}
\section{Charte de code}

\subsection{Formatage du code}
Les accolades sont positionnées en fin et début de fonction ou de bloc en 
respectant les conventions adoptées dans les environnements Visual Studio 
mais aussi www.gnu.org. Ce formatage permet de rendre le code plus aéré, 
et ainsi faciliter la relecture.

\begin{minted}{c}
static char *concat(char *s1, char *s2)
{
    ...
}
\end{minted}

Le code sera tabulé par bloc comme dans l'exemple suivant:
\begin{minted}{c}
void foo(int x)
{
    int res = 0;
    for (int i = 0; i < x; ++i)
    {
        res += i * x;
    }
    printf("res : %d\n", res);
}
\end{minted}

De plus les recommandations suivantes seront respectés:
\begin{itemize}
	\item Une seule instruction par ligne.
	\item Une seule déclaration par ligne.
	\item Une tabulation sera transformée en quatre espaces.	
	\item La longueur maximale d'une ligne est de 80 charactères.
    \item La définition des variables se fait au début d'une fonction avant
              toutes autres instructions. 
\end{itemize}

Enfin les opérandes utilisées pour les opérations \mintinline{c}{&&} et \mintinline{c}{||}
seront entourées de parenthèses afin d'éviter les ambiguïtés :
\begin{minted}{c}
if ((x != 0) && (isValid(y) || y == 5))
{
    ...
}
\end{minted}

\subsection{Sortie de fonctions}

Un seul \mintinline{c}{return} par fonction afin de
\begin{itemize}
	\item Faciliter la trace de la sortie.
	\item Le placement de break points lors du debug.
	\item Faciliter l’analyse de la fonction.
\end{itemize}

Les \mintinline{c}{goto} sont interdits.

\subsection{Commentaires}
Les commentaires ne doivent pas qu’uniquement formuler en texte ce qui est traduit
dans le langage de programmation (qui se doit d'être lisible) : \newline
Ils apportent une réelle plus-value dans la compréhension du programme ou doivent
attirer l’attention du mainteneur sur certains choix réalisés. Les commentaires seront 
rédigés en anglais. \newline
Un commentaire doit être sur une seule ligne (ne peut pas être sur la même ligne 
que du code).
 
\subsection{Nommages}
La définition des constantes seront en majuscules séparé par des underscores.\newline
Pour la définition d'un type, d'une structure ou d'une classe la notation 'PascalCase'
sera utilisé: 
\begin{minted}{c}
typedef struct 
{
    ...
}
SuperComplexDataType;
\end{minted}

Pour les variables et les fonctions les noms devront être significatifs et 
pertinents.\newline
Ils sont constitués uniquement des caractères non accentués suivants: 
[a-z], [A-Z], [0-9]\newline
Avec les restrictions suivantes :
\begin{itemize}
	\item Le nom ne doit pas commencer par un chiffre.
	\item Le nom ne peut pas avoir de caractères "underscore".
	\item Les noms doivent être en 'camelCase', en Anglais.
	\item Les noms ne peuvent pas commencer par des majuscules exception faite
          des acronymes (voir plus bas).
\end{itemize}
Les noms doivent être homogènes pour un domaine fonctionnel donné. Par exemple, 
toutes les fonctions relatives à la gestion de la mémoire dans une API peuvent 
être du type: \newline 
\begin{verbatim}
APIMemoryOpen(), APIMemoryClose(), APIMemorySet(...)
\end{verbatim}



\section{API}
\subsection{Script de génération}
\label{api:gen}
\begin{minted}{c}
// Récupère les droits d'une licence à partir d'un fichier de
// licence, renseigné par son chemin.
char *hashWithSHA256(const char *string);

// Génère une signature de dataToSign à partir de la clé privé asymétrique 
//  privateKey. 
char *generateSignature(const char *dataToSign, const char *privateKey);
  
// Encode une chaîne de caractères en base64
char *encodeBase64(const char *string);
\end{minted}

\subsection{Système de vérification - Librairie}
\label{api:lib}
\inputminted{csharp}{lib.cs}

