\documentclass{article}

\usepackage[utf8]{inputenc}
\usepackage[margin=3.8cm]{geometry}
\usepackage{hyperref}
\usepackage{minted}
\usepackage[french]{babel}
\usepackage[T1]{fontenc}

\title{
    \Huge
    Fiches techniques\\
    Génération de certificatsS
}
\author{\huge Noé Dallet}
\date{\huge \today}


\begin{document}

\maketitle
\newpage

\section{Personnes}

Dans la suite de ce document, quelques "personnes" entrent en jeu, et peuvent nécessiter une clarification préalable:

\begin{itemize}
    \item \textbf{l'autorité} : c'est un agent de confiance externe qui s'assure que les communications en interne se font comme il faut, et de manière sécurisé.
    \item \textbf{l'utilisateur} : c'est la personne qui souhaites se procurer le logiciel de gestion des licences, un logiciel protégé et sa licence. 
    \item \textbf{le fournisseur du service} : c'est la "personne" (c'est plutôt un PC/serveur) qui fournit à l'utilisateur le logiciel de gestion des licences, un logiciel protégé et qui génère la licence de l'utilisateur.
    \item \textbf{le pirate} : il intervient lorsqu'il trouve une faille, son but est d'abuser le système.
\end{itemize}


\section{Problématique}

Dans le cadre de ce projet, le client demande d'avoir une autorité capable de délivrer des licences, et un utilisateur qui exécute un programme qui communique des informations sensibles avec cette autorité.

Il faut être en mesure de certifier que chacun des deux partis est bien la personne qu'il prétends être.

Dans le cas contraire plusieurs attaques peuvent être faite à l'encontre du système :

\begin{itemize}
    \item L'utilisateur demande une clef à l'autorité, et, par un moyen ou un autre, un pirate s'empare de la connexion de l'utilisateur et se fait passer pour lui (vu que la connexion n'est pas protégée). Alors le serveur pense parler à l'utilisateur, et le pirate envois alors ses "informations matériel" pour générer sa licence. La licence est alors dépendante du matériel du pirate, et pas de l'utilisateur qui en a fait la demande, le pirate a donc volé une licence.
    \item L'utilisateur télécharge son programme, mais en réalité, un pirate se fait passer pour l'autorité, qui est aussi en charge de fournir les programmes. Par exemple celui qui gère les licences. Alors le pirate est en mesure de fournir à l'utilisateur un programme de gestion des licences modifié, et donc sûrement malveillant (il pourrait par exemple voler ses licences comme dans l'attaque précédente.
\end{itemize}

Une deuxième problématique entre en jeu : l'utilisateur lui même, peut essayer de dupliquer sa licence pour la donner à un ami; ici on supposera que cette attaque est impossible car les licences sont censés être dépendantes du matériel.\\
De plus elle ne rentre pas dans le cadre de cette fiche technique ci.

\section{Solution}

Pour être en mesure de certifier l'authenticité des deux partis, on utiliseras des certificats de type "x.509" qui repose sur un chiffrement asynchrone, on a donc besoin d'un couple de certificats à chaque fois (un publique et un privé). Dans ce système pour certifier l'authenticité d'un parti, une autorité de confiance doit "donner son avis" sur ce parti, plus précisément : l'autorité génères un certificat à partir de son certificat privé. Tout certificat publique pouvant être décrypté avec le certificat publique de l'autorité est donc de confiance. \\
Dans notre cas on a besoin : 
\begin{enumerate}
    \item d'un couple de certificats SSL, au moins pour distribuer le logiciel de gestion des licences sur un serveur web via https (un certificat SSL est un certificat x.509, qui est fait exprès pour se connecter sur un serveur https), délivré par une autorité de certification dite "racine" (que tout le monde connaît, et à qui tout le monde fait confiance, car on sait qu'elle est obligatoirement authentique).\\ . Quelques-unes de ces autorités sont gratuites (comme "letsencrypt").
    \item il faut un couple de certificats x.509 pour les échanges entre l'application de gestion des licences et le fournisseur du service. Ce couple de certificat peut être certifié par le fournisseur du service directement. Puisque le certifier par une autorité de certification n'apporterais ici aucun gain en sécurité.
\end{enumerate}

Pour ce faire :

\begin{enumerate}
    \item \begin{minted}{bash}
    openssl req -newkey rsa:4096 -keyout pk.key -out CSR.csr \
        -subj '/CN=DOMAINE'
    \end{minted}
    génère une demande de certification à envoyer à une autorité racine (CSR : Certificate signing request) celle ci se chargeras de signer le certificat publique contenu dans le CSR. (pk.key est la clef privée)
    \item \begin{minted}{bash}
    openssl req -x509 -newkey rsa:4096 -keyout pk.key -out cert.pem \
        -sha256 -subj '/CN=DOMAINE' -days NOMBRE
    \end{minted}
    génère un couple de certificats (cert.pem est publique et pk.key privé) sur le domaine DOMAINE valable NOMBRE jours et auto-certifié.
\end{enumerate}

\subsection{Certifier le fournisseur du service}

Le mécanisme se découpe en six étapes :
\begin{enumerate}
    \item Le fournisseur du service effectues une demande de certificat incluant une clef publique fraîchement généré, à l'autorité racine
    \item L'autorité racine lui chiffre avec sa clef privée et lui envois le résultat
    \item Le fournisseur du service envoi sa clef publique chiffrée par l'autorité racine à un utilisateur
    \item L'utilisateur du service a déjà la clef publique de l'autorité racine, il l'utilise pour déchiffrer la clef publique du fournisseur du service.
    \item Le fournisseur du service envoi ses messages chiffrés avec sa clef privée
    \item L'utilisateur du service déchiffre les messages avec la clef de l'étape 4.
\end{enumerate}

Si le message est correctement déchiffré l'émetteur est assuré d'être celui qu'il prétends.


\subsection{Certifier l'utilisateur du service}

Ici pour certifier l'authenticité de l'utilisateur le même principe est mis en place mais un roulement est fait : le rôle de autorité racine est joué par le fournisseur du service, le rôle du fournisseur du service par l'utilisateur qui cherche à être vérifié, et le rôle de l'utilisateur par le fournisseur du service également.\\
Quésako ?\\
C'est simple, à la première connexion sur le site avec le PC utilisateur qui fait tourner les programmes sous licence, l'utilisateur effectues l'étape 1, 2, 3, 4 et 5.  Tant que le message se décrypte par le fournisseur du service sans nouvelle clef, le PC de l'utilisateur n'as pas changé (ou alors sa clef privée a fuité). Si l'utilisateur change de machine ses données d'authentification sont redemandés, si elles sont correctes, c'est la bonne personne (ou alors elles ont fuité), alors on ajoute le PC à la liste de confiance du fournisseur et on re-fait tourner les étapes 1 à 5.

De cette façon on s'assure que si l'utilisateur fait attention à ne pas faire fuiter ses données, alors il est protégé.

\end{document}


