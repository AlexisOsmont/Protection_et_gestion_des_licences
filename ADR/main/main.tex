\chapter{Terminologies}

\begin{itemize}
	\item Le client est le commanditaire du projet.
	\item Un utilisateur est une personne souhaitant utiliser un logiciel du client. 
	\item Une licence est un droit accordé pour une machine et un utilisateur d'utiliser un logiciel donné.
	\item Craquer un logiciel est le fait de pouvoir l'utiliser sans avoir payé pour son utilisation. 
	Soit en modifiant le code compilé, soit en utilisant une autre méthode. 
\end{itemize}

\chapter{Liste des points durs}

\begin{itemize}
	\item \textbf{Minimisation des risques de failles d'implémentations} \newline
	Étant donné notre manque d'expérience dans les domaines des gestionnaires de licences
	et de la greffe de code, notre incapacité à la minimisation des failles d'implémentation est un risque élevé.\newline
	
	\item \textbf{Programmation de l'algorithme de signature ElGamal} \newline
	L'un des points les plus importants est la signature, la principale difficultée ici est de créer une signature unique grâce à 
	l'algorithme de signature ElGamal que nous devront coder "à la main" afin d'alléger au maximum l'exécutable de verification de licence.	\newline
	
	\item \textbf{Obfuscation du code} \newline
	L'obsfuscation de code est un élément essentiel car le but de notre projet est de protéger des logiciels du cracking. 
	Le risque majeur ici est de ne pas suffisamment obfusquer le code et donc de le rendre facilement accessible.\newline
	
	\item \textbf{Utilisation du C pour une DLL ou développement} \newline
	L'utilisation du langage C est l'un de nos choix en remplacement du C++ et C\# même si celui-ci possède moins d'utilitaires 
	que les autres, et est par conséquent plus assujetti aux failles d'implémentation.\newline
	
	\item \textbf{Injection de Code dans un PE} \newline
	L'injection de code peut, si elle est mal implémentée laisser des failles facilement utilisables pour des personnes malintentionnées.\newline
	
	\item \textbf{Respect des échéances} \newline
	Nous avons pris du retard dans la production de tous les documents demandés, dû à des délais de remise de travail demandé ainsi qu'aux 
	difficultées rencontrées quant à l'organisation de réunion entre membres du projet.\newline
\end{itemize}

\chapter{Evalutaion du contexte}

\section{Particularités du sujet}
Au début du projet, nous n'avions pas eu tous les détails sur la nature du projet. Cela nous a donc ralentit
dans notre compréhension du sujet et sur la production de documents.
Nous possédons des logiciels de référence sur lesquels se baser.\\ \newline

Nous sommes face à un sujet pluridisciplinaire, plusieurs grosses parties seront à 
développer et notre experience dans chaque domaine est fluctuante.
\begin{itemize}
	\item Application client serveur.
	\item Interface Web.
	\item Logiciel client Windows.
\end{itemize}

\section{Définition du besoin}

Le produit final sera réellement utlisé par le client ce qui rajoute une pression supplémentaire concernant la qualité 
du rendu final et l'éventail des fonctionnalitées à implémenter. Cela a tout de même le bénéfice d'avoir un référent 
technique très impliqué (notre client est aussi notre référent technique).
\begin{itemize}
	\item Attente de la part du client
\end{itemize}

Le produit final sera utilisé par des personnes non formées à l'informatique. 
\begin{itemize}	
	\item Les interfaces devront être pensées pour en facilité l'usage au près des utilisateurs.\newline
\end{itemize}

\section{Disponibilité des acteurs et ressources}
Les acteurs
\begin{itemize}
	\item Difficulté à organiser des réunions entre les membres du groupe.
	\item Réunion avec le professeur de gestion de projet 1 fois toutes les 2 semaines. 
	\item Réunion avec le client 1 fois toutes les 2 semaines.\newline
\end{itemize}

Les ressources
\begin{itemize}
	\item Beacoup de documentation sur internet
	\item Logiciels de référence intellilock, etc..
	\item Référent technique seulement 1 fois toutes les 2 semaines...
	\item Professeurs de crypto disponnibles.\newline
\end{itemize}

\section{Composition de l'equipe}
Nous nous connaissions déjà et communication facile lors des réunions.

\section{Connaissances techniques personnelles}
\begin{itemize}
	\item Obfuscation : Aucune expérience.\newline
	\item Injection de code : Aucune expérience.\newline
	\item Application client serveur : Alexis a le plus d'expérience dû à son stage qui a consisté en la réalisation 
		  	d'ajout de fonctionnalitées à un projet utilisant le framework Symfony.\newline
	\item Interface Web : Experience homogène dans le groupe. (Apprentissage des cours de Licence Informatique).\newline
	\item Développement logiciel Windows : Noe a le plus d'expérience dû à son stage qui a consisté en la programmation en 
			.NET de logiciels de gestion et de configuration d'un laser industriel.\newline
	\item Implémentation d'algorithme de cryptographie : Aucune expérience.
\end{itemize}

\section{La complexité des solutions techniques}

\begin{itemize}
	\item Obfuscation de code
	\item API Rest et utilisation depuis le logiciel de vérification de licences
	\item Injection de code et Signature ElGamal
	\item Interfaçage C / C\#
\end{itemize}

\section{Perturbations engendrées par les autres activités}
\begin{itemize}
	\item Examens
	\item Autres Projets
	\item Travaux personnels (TP)
\end{itemize}

\chapter{Ordonner les points durs}
\begin{enumerate}
	\item Injection de Code dans un PE
	\item Minimisation des risques de failles d'implémentation
	\item Obfuscation du code
	\item Gestion du temps
	\item Algorithme de signature ElGamal
	\item Utilisation du C et interfaçage avec du C\# 
\end{enumerate}

\chapter{Definitions des risques associés}

\begin{description}

    \item[Injection de Code dans un PE]\
	Dans le cas d'un échec de l'injection de code dans un exécutable windows, nous ne pourrions pas greffer dynamiquement
	notre code de vérification de licence dans les programmes produits par le client. Cela lui demanderait donc un travail 
	supplémentaire sur ses logiciels, et les demandes du client ne seront donc pas totalement satisfaite.
	À savoir que le client dans le cadre de ce projet ne demande à minima qu'une preuve de concept de l'injection de code.
	Dans l'espoir que les étudiant reprenant ce projet ultérieurement soient déjà avancés.
		
        \begin{description}
            \item[Algorithme de signature ElGamal]
			Cette étape est critique dans le bon déroulement de la phase d'injection de code. Le code à injecter se doit d'être 
			le plus léger possible. Ainsi il ne nous faut faire appel à aucune DLL exterieure et coder un algorithme de siganture de nous
			même est une solution qui nous a été proposé par le client. Ne pas valider cette étape est probable étant donné
			la complexité de la tâche, néanmmoins l'impacte sur le projet n'est pas critique; L'utilisation d'une DLL restant
			toujours possible malgré le coût engendré.\newline
		\end{description}

    \item[Minimisation des risques de failles d'implémentation]
	À la fois dans un but pédagogique et parceque le client utilisera réellement le fruit de ce projet, nous devrons
	faire au maximum afin de minimiser les risques de failles causées par de potentielles erreures d'implémentation de notre part.
	Dans le cas contraire, la solution fournit au client se verrait alors inutile, ne pouvant pas protéger efficacement les logiciels
	de ce dernier. 
	Pour ce faire il nous faudra mettre en place un système de test unitaires, et de validation des ajouts et des modifications au code.
	Afin de garantir qu'à chaque étape du projet tous les contrats initialements définits soient bien respectés.\newline

\item[Obfuscation du code]
	Les risques associés au manque d'obfuscation du code rejoignent ceux du point précédent. Le produit final finira par être trop facilement
	détournable et le produit se verra d'efficacité limitée. Néanmoins le niveau d'expertise nécessaire à la compréhension d'un code décompilé 
	en assembleur est suffisamment élevé pour estimer que les risques qu'un des utilisateurs y parvienne sont faibles; Étant donné que le public
	visé par notre produit et le client, correspond à des utilisateurs non initiés à l'informatique. Ce faisant nous pouvons estimer la criticité 
	des conséquences associées à ce risque comme non critique.\newline

\item[Utilisation du C et interfaçage avec du C\#]
	Globalement nous n'avons que peu d'expéricence dans le domaine du développement windows. Ainsi pouvoir se rattacher à un langage que nous
	connaissons déjà pourrait accélérer le développement. Nous avons donc choisi d'utiliser du C lors de ce projet, or nous devront créer une 
	interface compatible avec du C\#, notamment afin de créer une DLL utilisable par d'autre développeur afin de sécuriser leurs logiciels.
	Ici repose la difficultée. Les chances d'un échec sur ce sujet son minime au vu de la complexité de la tache, mais l'impacte serait très fort.
	Cela nous forcerait à revoir notre stratégie de développement et par conséquent de perdre du temps par rapport à notre planing.

\end{description}

\chapter{Top cinq des risques (Bilan)}

\begin{table}[!h]
    \small
    \begin{tabular}{|m{2.5cm}|m{2.5cm}|m{2.5cm}|m{1.5cm}|m{1.5cm}|m{4cm}|} 
	\hline
	\textbf{Risque} & \textbf{Craintes} & \textbf{Effets} & \textbf{Impacts} & \textbf{Prob.} & \textbf{Stratégie}\\
	\hline
	Failles d'implémentation 
		& Cracking du logiciel et de son injection 
		& Perte de controle du logicel et Insatisfaction du client 
		& Critique 
		& Forte 
		& Effectuer des tests unitaires sur les logiciels et les exécutables.\\
	\hline
	Échec de l'injection de code 
		& Que la complexité de la tâche soit trop élevée et que l'objecif ne soit pas atteint 
		& Insatisfaction client / Perte de temps et d'énergie 
		& Modéré 
		& Très forte 
		& Effectuer une preuve de concept avant le début de la phase de développement.\\
	\hline
	Signature ElGamal 
		& Complications lors de l'accès au logicel 
		& Impossible d'utliser le logicel / Objectif injection de code non atteint 
		& Fort pour l'injection 
		& Forte 
		& Effectuer une preuve de concept au plus tôt.\\
	\hline
	Obfuscation du code 
		& Cracking du logiciel et de son injection 
		& Perte de controle du logicel d'injection de code 
		& Important 
		& Très forte 
		& Effectuer des tests unitaire sur les logiciels et les exécutables.\\
	\hline
	Respect des échéances 
		& Manque de temps pour un rendu complet 
		& Prise de retard par rapport aux TP de Gestion de Projet
		& Important 
		& Forte 
		& S’organiser et séparer efficacement les tâches pour bien gérer le travail.\\
	\hline	    
    \end{tabular}
\end{table}

