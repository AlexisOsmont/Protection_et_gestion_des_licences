\chapter{Terminologies}

\begin{itemize}
	\item Le client est le commanditaire du projet.
	\item Un utilisateur est une personne souhaitant utiliser un logiciel du client. 
	\item Une licence est un droit accordé pour une machine et un utilisateur d'utiliser un logiciel donné.
	\item Craquer un logiciel est le fait de pouvoir l'utiliser sans avoir payé pour son utilisation. Soit en modifiant le code compilé, soit en utilisant une autre méthode. 
\end{itemize}

\chapter{Liste des points durs}

\begin{itemize}
	\item \textbf{Minimisation des risques de failles d'implémentations} \newline
	Étant donnée notre manque d'expérience dans les domaines des gestionnaires de licences
	et de la greffe de code, notre incapacité à la minimisation des failles d'implémentation est un risque élevé.\newline
	
	\item \textbf{Programmation de l'algorithme de signature ElGamal} \newline
	L'un des points les plus importants est la signature, la principale difficultée ici est de créer une signature unique grâce à l'algorithme de signature ElGamal que nous devront coder "à la main" afin d'alléger au maximum l'exécutable de verification de licence.	\newline
	
	\item \textbf{Obfuscation du code} \newline
	L'obsfuscation de code est un élément essentiel car le but de notre projet est de protéger des logiciels du cracking. Le risque majeur ici est de ne pas suffisamment obfusquer le code et donc de le rendre facilement accessible.\newline
	
	\item \textbf{Utilisation du C pour une DLL ou développement} \newline
	L'utilisation du langage C est l'un de nos choix en remplacement du C++ et C\# même si celui-ci possède moins d'utilitaires que les autres, et est par conséquent plus assujetti aux failles d'implémentation.\newline
	
	\item \textbf{Injection de Code dans un PE} \newline
	L'injection de code peut, si elle est mal implémentée laisser des failles facilement utilisables pour des personnes malintentionnées.\newline
	
	\item \textbf{Le temps} \newline
	Nous avons pris du retard, dû à des délais de remise de travail demandé ainsi qu'aux 
	difficultées rencontrées quant à l'organisation de réunion entre membres du projet.\newline
\end{itemize}

\chapter{Evalutaion du contexte}

\section{Particularités du sujet}
Au début du projet, nous n'avions pas eu tous les détails sur la nature du projet. Cela nous a donc ralentit
dans notre compréhension du sujet et sur la production de documents.
Nous possédons des logiciels de référence sur lesquels se baser.\\ \newline

Plusieurs parties à réaliser, sujet pluridisciplinaire :
\begin{itemize}
	\item Programmation Serveur.
	\item Interface Web.
	\item Logiciel client Windows.
\end{itemize}

\section{Définition du besoin}
Le produit final sera utilisé par des personnes non formées à l'informatique. 
\begin{itemize}	
	\item Les interfaces devront être pensées pour en facilité l'usage au près des utilisateurs.\newline
\end{itemize}

Le produit final sera réellement utlisé par le client ce qui rajoute une pression supplémentaire concernant la qualité du rendu final et l'éventail des fonctionnalitées à implémenter. Cela a tout de même le bénéfice d'avoir un référent technique très impliqué (notre client est aussi notre référent technique).
\begin{itemize}
	\item Attente de la part du client
\end{itemize}

\section{Disponibilité des acteurs et ressources}
Les acteurs
\begin{itemize}
	\item Difficulté à organiser des réunions entre les membres du groupe.
	\item Réunion avec le professeur de gestion de projet 1 fois toutes les 2 semaines. 
	\item Réunion avec le client 1 fois toutes les 2 semaines.\newline
\end{itemize}

Les ressources
\begin{itemize}
	\item Beacoup de documentation sur internet
	\item Logiciels de référence intellilock, etc..
	\item Référent technique seulement 1 fois toutes les 2 semaines...
	\item Professeurs de crypto disponnibles.\newline
\end{itemize}

\section{Composition de l'equipe}
Nous nous connaissions déjà et communication facile lors des réunions.

\section{Connaissances techniques personnelles}
\begin{itemize}
	\item Obfuscation : Aucune expérience.
	\item Injection de code : Aucune expérience.
	\item Programmation Serveur : Sami a le plus d'expérience dû à son stage.
	\item Interface Web : Experience homogène dans le groupe. (Apprentissage des cours de Licence Informatique).
	\item Développement logiciel Windows : Noe a le plus d'expérience dû à son stage.
	\item Implémentation d'algorithme de cryptographie : Aucune expérience. 
\end{itemize}

\section{La complexité des solutions techniques}
Les solutions techniques n'étant pas encore vraiment décidées.
\begin{itemize}
	\item ElGamal.
\end{itemize}

\section{Perturbations engendrées par les autres activités}
\begin{itemize}
	\item Examens
	\item Autres Projets
	\item Travaux personnels (TP)
\end{itemize}

\chapter{Ordonner les points durs}
\begin{itemize}
	\item 1 - Injection de Code dans un PE
	\item 2 - Minimisation des risques de failles d'implémentation
	\item 3 - Obfuscation du code
	\item 4 - Gestion du temps
	\item 5 - Algorithme de signature ElGamal
	\item 6 - Utilisation du C et interfaçage avec du C# 
\end{itemize}

\chapter{Definitions des risques associés}

\chapter{Top cinq des risques (Bilan)}

\begin{table}[!h]
    \small
    \begin{tabular}{|m{2.5cm}|m{2.5cm}|m{2.5cm}|m{1.5cm}|m{1.5cm}|m{4cm}|} 
	\hline
	\textbf{Risque} & \textbf{Craintes} & \textbf{Effets} & \textbf{Impacts} & \textbf{freq} & \textbf{Stratégie}\\
	\hline
	Failles d'implémentation & Cracking du logiciel et de son injection &  Perte de controle du logicel & Critique & Trés forte & Effectuer des testes unitaire sur les logiciels et les exécutables.\\
	\hline
	Injection de Code & Cracking du logiciel et de son injection &  Perte de controle du logicel / Insatisfaction du client & Critique & Trés forte & Effectuer des testes unitaire sur les logiciels et les exécutables.\\
	\hline
	Obfuscation du code & Cracking du logiciel et de son injection &  Perte de controle du logicel d'injection de code & Important & Trés forte & Effectuer des testes unitaire sur les logiciels et les exécutables.\\
	\hline
	Gestion du temps & Manque de temps pour un rendu complet &  Prise de retard par rapport aux TP de Gestion de Projet & Important & forte & S’organiser et séparer efficacement les tâches pour bien gérer le travail.\\
	\hline
	Signature ElGamal & Complications lors de l'accès au logicel &  Impossible d'utliser le logicel & Fort & Faible & Effectuer des testes unitaire sur les signatures pour prevenir ce risque.\\
	\hline	    
    \end{tabular}
\end{table}

