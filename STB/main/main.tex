\chapter{Objectifs}

Le but de ce projet est de fournir une solution permettant de proteger les 
applications du client sur Windows en évitant leur copie ou leur utilisation illégale. 
Il devra également permettre au client de générer et gérer des licences d'utilisation 
via une interface graphique. Il pourra ainsi définir des contraintes sur les licences 
comme par exemple la durée de validité. \newline
Cette outil vient en remplacement de solution déjà existante tel que 
	IntellilLock ou ElecKey, a l'instar de ces outils la prise en compte des paiements sera effectué par une solution externe du client.

\chapter{terminologies}

\begin{itemize}
	\item Le client est le commanditaire du projet.
	\item Un utilisateur est un client du client. 
	\item Une licence est un droit accordé pour une machine et un utilisateur d'utilisé un logiciel donné.
\end{itemize}

\chapter{Exigences}

\section{Developper une platforme simple pour permettre au utilisateur de s'enregistrer}
Cette platforme devra fournir une interface simple, facile d'accès et lisible notamment 
pour des personnes sans compétences en informatiques. Il faudra donc que l'outil dispose:
\begin{itemize}
	\item Un guide du site avec un tutoriel pour s'enregister et créer une licence.
	\item Une page de contact pour demander de l'aide.
	\item Une charte graphique épuré en suivant les recommandations matérial design.
\end{itemize}

\section{Developper un outil permettant de vérifier une licence}
Cette outil devra permettre de vérifier la validité et de l'authenticité d'une licence pour un logiciel. \newline
Dans un premier temps l'outil devra être intégré par le client lors du developpement de son application 
puis dans un second temps il pourra être directement ajouté au logiciel sans avoir besoin de modifier le 
code source.

\section{Developper une platforme pour permettre au client de manipuler / gérer les licences}
Cette platforme devra permettre au client de donner les autorisations nécessaires à un profil d'un utilisateur afin que celui ci puisse par la suite générer une licence. Pour cela il pourra accéder à un panneau de contrôle avec les options suivantes:
\begin{itemize}
	\item La liste des profils des utilisateurs.
	\item Une liste des nouveaux utilisateurs en attente de validation.
	\item La possibilité de consulter la les licences de chaque utilisateurs.
	\item La possibilité de définir la porté d'une licence pour un utilisateur 
				comme la durée de validité (potentiellement infini) ou des restrictions.
\end{itemize}   

\section{Exigences de Sécurité}
Les principaux points liés aux aspects de la sécurité sont les suivants:
\begin{itemize}
	\item Se protéger du reverse engineering via l'obfuscation du code de l'application.
	\item Les échanges de données sur le réseau seront chiffrés.
	\item Le stockage des données persistante sera lui aussi protégées.
\end{itemize}   

\chapter{Spécifications fonctionnelles de l'outil}

\section{Partie utilisateur}

Cette section décrit 
\subsection{inscription d'un utilisateur}

\subsection{generation de la licence}

\subsection{utilisation du logiciel avec la licence}

\section{Partie Client}

\subsection{activation d'un compte et gestion des droits}
