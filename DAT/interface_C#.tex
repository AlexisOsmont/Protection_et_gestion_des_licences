\documentclass{article}

\usepackage[margin=3cm]{geometry}
\usepackage[french]{babel}
\usepackage[utf8]{inputenc}
\usepackage[T1]{fontenc}
\usepackage{minted}

\begin{document}
\section{Bibliothèque C\#}
\subsection{Interface de la bibliothèque C\#}
\begin{minted}{csharp}
/*
 * Récupère l'identifiant de la carte réseau.
 */
public String getNetworkAdapterNo();

/*
 * Récupère le numéro de série du BIOS.
 */
public String getBIOSSerialNo();

/*
 * Récupère le numéro de série du disque.
 */
public String getDiskSerialNo();

/*
 * Récupère les droits d'une licence à partir d'un fichier de
 * licence, renseigné par son chemin.
 */
public String getRights(String pathToLicenseFile);

/*
 * Vérifie la signature à partir des informations du système,
 * de l'identifiant du logiciel, situé dans le programme, et des
 * droits, obtenus à partir du fichier de licence.
 */
public int verifySignature(String pathToLicenseFile);

/*
 * Génère un hashé à partir d'une chaîne de caractères donné
 * en entrée, en utilisant l'algorithme SHA-256.
 */
public String hashWithSHA256(String s);

/*
 * Décode une chaîne de caractères encodée en base64.
 */
public String decodeBase64(String s);
\end{minted}

\subsection{Contraintes}
Dans un premier temps, les fonctions de l'API pourront être implémentées en utilisant des bibliothèques systèmes
comme par exemple pour les informations hardware la classe \verb:System.Management:, ou bien la classe
\verb:System.Security.Cryptography: pour le hash avec SHA-256.\\

Dans un second temps, nous devrons implanter au maximum les fonctions nous-mêmes et les optimisés dans le but
de pouvoir intégrer ces fonctions directement dans un exécutable (greffe dans un PE) sans pour autant alourdir
ou ralentir l'éxécution de l'application.
\end{document}
