\documentclass{article}
\usepackage[utf8]{inputenc}
\usepackage{hyperref}
\usepackage{fontawesome}
\usepackage[margin=3.8cm]{geometry}
\usepackage[french]{babel}
\usepackage[backend=bibtex8]{biblatex}

\title{
    \Huge
    Compte-rendu\\
    Réunion numéro 1\\
}
\date{\huge 26 Octobre 2021}
\author{\huge Alexis Osmont\\}

\newcommand{\arrow}{$\rightarrow$ }

\begin{document}

\maketitle
\vspace{5cm}
    Présent à la réunion :
    \begin{description}
        \item Alexis Osmont
        \item Kaci Hammoudi
        \item Louka  Boivin
        \item Sami Babigeon
    \end{description}
\newpage

\section{Sujet de réunion}                                                                                 

Le but de cette réunion était de définir les besoins du client et de parler de la structure du projet pour donner une direction au projet.\\

Nous avons aussi abordé le sujet du document des spécifications technique des besoins (STB), et fait une présentation et une création de schémas au cours de la réunion.
\vspace{0.5cm}

\section{Retour sur compte rendu}
\subsection{STB}

\begin{itemize}
    \item Définition du besoin à cause de la faible documentation du sujet.
    \item Définition des technologies à utiliser.
    \item Définition des ordres de priorité des tâches.
    \item vérifications de conprenssion du sujet.
    \item Démonstration de logiciel proposant les même service (LOCKey...).
\end{itemize}

\subsection{Gestion de projet}
\begin{itemize}
    \item Règles de gestion de projet.
    \item Le projet est avant tout créé pour la gestion de projet.
    \item Mise en commun des connaissance.
    \item Choix de la methode de projet.

\end{itemize}
\section{Discussions d'aspects technique}

Certains points techniques ont été abordés durant cette réunion, tel que les signatures et les certificats, l'importance de
et l'aspect sécurité des licences.\\

Un element essentiel est la génération des licences et la récupération de données uniques qui permettent de creer une licence unique et le fait qu'elle ne soit pas réutilisable  
ni falsifiable.\\

Nous avons créé un schéma global de l'application et des technolgies qu'elle va embarquer, comme SQL, C ou C\#, apache pour le serveur et 
du bash pour l'injection de code\\

Plusieurs cas de génération d'utilisation de l'application ont été donnés au cours de la réunion pour permettre une comprehension totale du projet.
Un cas du coté utilisateur a été abordé ainsi qu'un cas du coté administrateur. Un exemple sur un logiciel concurrent déjà existant a été fait pour faire une liste
des demandes et des options des logicels tel que les options de licence et leurs gestionnaires par exemple.
\newpage


\end{document}
