\documentclass{article}
\usepackage[utf8]{inputenc}
\usepackage{hyperref}
\usepackage{fontawesome}
\usepackage[margin=3.8cm]{geometry}
\usepackage[french]{babel}
\usepackage[backend=bibtex8]{biblatex}

\title{
    \Huge
    Compte-rendu\\
    Réunion numéro 3\\
}
\date{\huge 7 Decembre 2021}
\author{\huge Alexis Osmont\\}

\newcommand{\arrow}{$\rightarrow$ }

\begin{document}

\maketitle
\vspace{5cm}
    Présent à la réunion :
    \begin{description}
        \item Alexis Osmont
        \item Kaci Hammoudi
        \item Louka  Boivin
        \item Noé Dallet
        \item Sami Babigeon
    \end{description}
\newpage

\section{Sujet de réunion}   

Le but de cette réunion était de présenter puis d'avoir un retour sur le document d'architecture technique (DAT) ainsi que sur le document d'analyse des risques (ADR) pour attester de la bonne direction du projet.\\

Nous avons aussi abordé le sujet du document d'architecture technique (DAT), des risques engendrés par un projet tel que celui ci et nous avont fait une présentation des autres documents à venir 
pour trouver des points à aborer sur l'ADR et le cahier de recette (CDR).
\vspace{0.5cm}

\section{Retour sur compte rendu}
\subsection{DAT}

\begin{itemize}
    \item Bonne structure. 
    \item Correspond aux besoins.
    \item Fautes d'orthographe.
    \item Convention sur la forme.
    \item 5.1.1 rôle : Serveur -> application (web).
    \item Framework à utiliser pour la MVC : codeIgniter.
    \item Spécifier ou modifier sur le schema API : ajouter la sécu d'authentification clairement(droit de lancer l'API).
    \item Ajout hardwareID Figure 5.3.
    \item Fonctions de protection dans le framework.
    \item Revoir la forme du model page 18.
    \item Spécification de l'API physique. 
    \item schema général bon
\end{itemize}

\subsection{ADR}

\begin{itemize}
    \item La liste des points durs : rien à dire.
    \item Bonne definition et ordre des points.
    \item 3.1 : Application client serveur -> programmation serveur.
    \item Précision sur les connaissances techniques (3.5).
    \item Complexité des solutions techniques : tout les points dur et aucune experience.
    \item 3.6 : Ajout obfuscation aussi.
    \item Précision des points 3.7 dire ce que ca engendre. 
    \item Revoir la mise en forme de certaines parties
    \item Revoir l'ordre du tableau en fin de doc
\end{itemize}

\subsection{PDD}

\begin{itemize}
    \item Pas le temps de faire une review de ce document.
\end{itemize}

\section{Discussions d'aspects technique}
Certains points techniques ont été abordés durant cette réunion, tel que la gestion des API et leurs sécurités, l'importance de
l'utilisation d'un framework simplifiant et structurant le travail. Le framework codeIgniter sera donc utilisé, de plus il
possede nativement des fonctions de sécurisation.
CodeIgniter peut permettre une mise en place serveur simplifié par exemple.
\\

La partie API et les fonctions principales ont été le points principales de la réunion
la simplicité, leurs parametrages et leurs utilités a été precisé et redifinie pour certaine.
Un seul controller sera mit en place.La partie 5 a été validé dans son ensemble il ne reste que des details de spécification à revoir.\\ \newline

La greffe de code ne doit pas obligatoirement être implementer en sa totalité, un exemple et le minimum requis
lors du projet et donc doit apparaitre sur les documents de besoin.

\newpage


\end{document}
