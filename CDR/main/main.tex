\chapter{Introduction}
Ce document présente les tests qui seront mis en place pour vérifier le bon fonctionnement et le
respect des exigences fonctionnelles et techniques de l'outil.
Ce document précise :
\begin{itemize}
    \item les conditions à satisfaire préalablement à l’exécution des tests,
    \item les moyens matériels requis (plate-forme de tests),
    \item la logique de leur déroulement (étapes successives),
    \item les conditions d’arrêt.
\end{itemize}
\medskip

Les fonctionnalités principales du logiciel (cas d'utilisations) sont les suivantes :
\begin{itemize}
    \item CA1 : Demande d'une licence,
    \item CA2 : Accord d'une licence,
    \item CA3 : Activation d'une licence,
    \item CA4 : Démarrage d'un logiciel protégé,
    \item CA5 : Création d'un compte sur le site Web,
    \item CA6 : Paramétrage d'une licence.
\end{itemize}
\medskip

Les différents modules à tester sont les suivants :
\begin{itemize}
    \item La plateforme Web (partie front-end et back-end),
    \item Le logiciel d'activation,
    \item Le système de vérification de licence.
\end{itemize}
\medskip

La partie interface et toute la logique lié à la plateforme sera bien évidemment à tester mais le
principal des tests seront sûr la sécurité de cette plateforme et de notre système de vérification
des licences.


\chapter{Documents de référence}
\textbf{Documents de spécification :}
\begin{itemize}
    \item Spécification technique des besoins
    \item Fiches techniques :
    \begin{itemize}
        \item Injection de code dans un PE
        \item Génération de licence
        \item Signature
        \item Obfuscation de code
        \item Sécurisation des bases de données
    \end{itemize}
    \item Document d'architecture technique
    \item Analyse des risques
\end{itemize}

\chapter{Terminologie}

\chapter{Environnement de test}

\section{Application Web}
La plateforme cible pour l'application web sera une machine virtuelle sur un serveur (Debian)
hébergé par l'université.

Pour être sûr de la validité de nos tests, tous les tests qui le pourront seront effectués
directement sur cette machine, ou un clone de la machine pour pouvoir revenir facilement à son
état initial, pour, par exemple, un test de déploiement de l'application.

L'API Rest de l'application web sera testé depuis nos machines depuis un système Linux et surtout
depuis un Windows car ce sera la plateforme cible pour notre outil.

\section{Logiciel d'activation}
Les tests concernant le logiciel d'activation comprendront les tests de l'API Rest et également
des tests par rapport à l'interface avec l'utilisateur. Le logiciel d'activation aura pour but
d'être utilisé sur Windows et le développement de son interface se fera à l'aide d'un outil Windows
(probablement en C\# avec Visual Studio).

Tous les tests concernant le logiciel d'activation se
feront donc sur Windows, de préférence sur plusieurs versions. Windows 10 et 11 pourront être
testés facilement, et selon si l'on dispose des machines necéssaires, Windows 7 et 8 pourront être
testés.

\section{Système de vérification}
Le système de vérification de licence devra être testé sur plusieurs systèmes. En effet, la
génération d'une licence sera effectué côté serveur, donc sur un système Linux, mais la
vérification, elle, sur un système Windows.

Il faudra donc testés les 2 cas, encore une fois sur plusieurs machines/versions.

\section{Outils de test}
Les outils que nous mettront en oeuvre pour nous aider à effectuer et gérer les tests sont les
suivants :
\begin{itemize}
    \item MantisBT,
    \item Validateur HTML/CSS de W3C.
\end{itemize}

\section{Jeu de données}
Nous définirons pour chacun des modules à tester un jeu de données de tests, sous forme de fichiers
CSV ou d'une base de donnée MySQL.
 
\chapter{Responsabilités}
Les responsabilités des tests seront d'abord réparties selon la répartiton des tâches, chacun
devra donc écrire des tests pour les modules qu'il développe et se créer ses propres données
de test.

Puis lors de la phase d'intégration, un journal de test sera régulierement mis à jour
afin de répertorier les bugs et les régressions de la solution et ainsi identifier plus      
facilement la source de l'erreur et en trouver l'auteur afin que celui-ci la résolve.


\chapter{Stratégie de tests}

\section{Démarche général et outils}
\label{section:dem_gen}
Nous avons choisi d'utiliser MantisBT (Mantis Bug Tracker) qui est un logiciel gratuit
et open source afin de répertorier les bugs, celui-ci permet d'enregistrer la déclaration
 d'un bug, et d'affecter quelqu'un à sa résolution. Ce dernier pourra alors rendre compte de
l'avancement de sa résolution, jusqu'à sa clôture. Le déclarant de l'anomalie peut  
s'informer à tout moment via le serveur Web de l'avancement du traitement de 
son problème.  

\section{Campagne de test}
Les tests seront lancés après chaque modification sur un module ou plusieurs modules, sur
le serveur de test.

\section{Tests de l'application web}

\subsection{Partie Frontend}

Afin de tester que l'interface graphique (vue du modèle MVC) fonctionne correctement, nous allons mettre en place les procédures suivantes:
\begin{itemize}

    \item vérifier la validité syntaxique des pages statiques en HTML / css via 
          le validateur du W3C. Celui ci peut être utilisé via l'interface web ou  
          être appelé directement via la commande curl:
          \begin{minted}{bash}
curl -H "Content-Type: text/html; charset=utf-8" \
    --data-binary @FILE.html \
    https://validator.w3.org/nu/?out=gnu
          \end{minted}
              
    \item vérification manuelle du rendu visuel de l'interface.
    \item tests sur différents navigateur (Chrome, Firefox, Opera).
            
\end{itemize}
\newpage

\subsection{Partie Backend}

Afin de tester que le modèle et le contrôleur du modèle MVC fonctionne correctement, 
nous allons mettre en place les procédures suivantes :
\begin{itemize}
    \item réalisation de tests unitaires sur le modèle et le contrôleur via 
          un script de test écrit en PHP, celui-ci pourra être lancer après 
          chaque mise à jour. Il utilisera les informations contenues dans la  
          base de données de test.
        
    \item test de l'API Rest du logiciel d'activation via un script bash 
          utilisant la commande curl avec le jeu de données de la base de test.
        
    \item test du programme de génération de licence via un script bash qui
          utiilisera les données de test.
\end{itemize}

\subsection{Sécurité}
Il faudra prêter une attention particulière à la sécurité de notre plateforme afin de garantir un système
de gestion et protection de licences fiable (voir Analyse des risques).

Par conséquent, afin d'en tester la robustesse nous allons effectuer différents types d'attaques sur notre site :
\begin{itemize}
    \item Injections XSS
    \item Injections SQL
    \item Tentatives d'accès non autorisées (par exemple : via l'URL)
    \item ... (voir OWASP - Top 10)
\end{itemize}


\section{Logiciel protégé}

Une fois le projet suffisamment avancé pour pouvoir commencé à générer des licences et protéger des logiciels
avec, il nous faudra procéder à des tests d'efficacité de celui-ci.

Également nous allons essayer d'obfusquer au maximum l'exécutable produit pour rendre le reverse engineering
plus compliqué pour un attaquant. Il faudra donc que l'on se mette à la place de cet attaquant en essayant
de retrouver des informations sur le logiciel en utilisant des techniques de reverse engineering.


\chapter{Gestion des anomalies}
Pour la gestion des anomalies nous utiliserons le logiciel MantisBT (comme précisé section
\ref{section:dem_gen}) afin de répertorier les tests et les anomalies qui en ressortent.\newline

Avant la phase d'intégration du projet, lorsque chacun aura une partie indépendante à développer,
la gestion des anomalies sera à la charge de chacun. Il sera cependant conseillé de consigner ses
avancements et les anomalies afin de permettre à un autre membre du groupe de reprendre cette
partie si besoin.\newline

Cependant une fois que nous aurons mis en commun les différentes parties de l'application, le
référencement des anomalies et la journalisation des tests via MantisBT sera obligatoire afin de
pouvoir identifier la source et l'auteur des anomalies.


\chapter{Procédure de test}

\section{Procédure de test du CA1}
\begin{table}[!h]
        \centering
        \begin{tabular}{|m{0.6cm}|
                         >{\raggedright\arraybackslash}m{4cm}|
                         >{\raggedright\arraybackslash}m{6.4cm}|
                         >{\raggedright\arraybackslash}m{2cm}|
                         m{1cm}|}
            \hline
            \multicolumn{3}{|c|}{
                \textbf{Objet testé: } Application serveur 
            } & \multicolumn{2}{|c|}{
                \textbf{Version: } version    
            } \\
            \hline
            \multicolumn{5}{|c|}{\textbf{Objectifs de test:} 
                vérification du CA1 : Demande d'une licence} \\
            \hline
            \multicolumn{5}{|c|}{
                \textbf{Procédure n°}1 - Test CA1
            } \\
            \hline
            N° & Actions & Résultats attendus & Exigences & OK / NOK \\
            \hline      % start of the array
            1 & Le testeur doit se rendre sur l'interface de l'application et 
                effectuer demande de licence (fonctionalité implémenté 
                sous la forme d'un bouton présent sur l'interface d'un
                utilisateur) via un compte utilisateur déjà existant. 
              & Le testeur doit constater un message lui indiquant le succès (ou non)
                de sa demande. En cas de succès, l'administrateur doit être notifié via 
                un mail de la demande et la base de données doit être mise à jour: 
                création d'une entrée dans la table licence comportant l'identifiant du
                logiciel demandé, de l'utilisateur ayant effectué la demande et avec le 
                statut "\emph{Pending}".
              & Exigences vérifiées & \\
            \hline
        \end{tabular} 
        \label{tab:tab1}
\end{table}
\newpage

\section{Procédure de test du CA2}
\begin{table}[!h]
        \centering
        \begin{tabular}{|m{0.6cm}|
                         >{\raggedright\arraybackslash}m{4cm}|
                         >{\raggedright\arraybackslash}m{6.4cm}|
                         >{\raggedright\arraybackslash}m{2cm}|
                         m{1cm}|}
            \hline
            \multicolumn{3}{|c|}{
                \textbf{Objet testé: } Application serveur
            } & \multicolumn{2}{|c|}{
                \textbf{Version: } version
            } \\
            \hline
            \multicolumn{5}{|c|}{\textbf{Objectifs de test:} 
                vérification du CA2 : Accord d'une licence }\\
            \hline
            \multicolumn{5}{|c|}{
                \textbf{Procédure n°}2 - Test CA2
            } \\
            \hline
            N° & Actions & Résultats attendus & Exigences & OK / NOK \\
            \hline      % start of the array
            1 & Le testeur doit se connecter à l'interface de l'application
                via le compte administrateur, aller dans le panneau de configuration
                de l'application, et sélectionner une entrée dans la table des demandes 
                en attentes puis cliquer sur le bouton "\emph{Accepter}". 
              & Le testeur doit constater un message lui indiquant le succès (ou non)
                de l'opération. En cas de succès, l'utilisateur qui a fait la demande 
                doit être notifié via un mail de l'acceptation de sa demande et la 
                base de données doit être mise à jour: mise à jour de l'entrée dans 
                la table licence comportant l'identifiant du logiciel demandé et de l'utilisateur 
                ayant effectué la demande et avec le statut "\emph{Accepted}".
              & Exigences vérifiées & \\
            \hline
            2 & Le testeur doit se connecter à l'interface de l'application
                via le compte administrateur, aller dans le panneau de configuration
                de l'application, et sélectionner une entrée dans la table des demandes 
                en attentes puis cliquer sur le bouton "\emph{Refuser}". 
              & Le testeur doit constater un message lui indiquant le succès (ou non)
                de l'opération. En cas de succès, l'utilisateur qui a fait la demande 
                doit être notifié via un mail du refus de sa demande et la 
                base de données doit être mise à jour: mise à jour de l'entrée dans 
                la table licence comportant l'identifiant du logiciel demandé et de l'utilisateur 
                ayant éfféctué la demande et avec le statut "\emph{Refused}".
              & Exigences vérifiées & \\
            \hline
        \end{tabular} 
        \label{tab:tab2}
\end{table}
\newpage

\section{Procédure de test du CA3}
\begin{table}[!h]
        \centering
        \begin{tabular}{|m{0.6cm}|
                         >{\raggedright\arraybackslash}m{4cm}|
                         >{\raggedright\arraybackslash}m{6.4cm}|
                         >{\raggedright\arraybackslash}m{2cm}|
                         m{1cm}|}
            \hline
            \multicolumn{3}{|c|}{
                \textbf{Objet testé: } Logiciel d'Activation 
            } & \multicolumn{2}{|c|}{
                \textbf{Version: } version    
            } \\
            \hline
            \multicolumn{5}{|c|}{\textbf{Objectifs de test:} 
                vérification du CA3 : Activation d'une licence} \\
            \hline
            \multicolumn{5}{|c|}{
                \textbf{Procédure n°}3 - Test CA3
            } \\
            \hline
            N° & Actions & Résultats attendus & Exigences & OK / NOK \\
            \hline      % start of the array
            1 & Le testeur doit lancer le logiciel d'activation sur la
                machine cible (et vérifier qu'il a accès à internet). 
              & Une fois démarré, le logiciel doit afficher un menu avec 
                deux entrées de texte comportant respectivement les champs
                "\emph{Identifiant}" et "\emph{Mot de passe}" ainsi qu'une liste
                déroulante contenant la liste de logiciels disponibles
                (cette liste doit correspondre à la liste présente dans
                 la base de données).
              & Exigences vérifiées & \\
            \hline
            2 & Le testeur doit réaliser l'action 1 puis séléctionner 
                un logiciel présent dans la liste, et saisir les 
                identifiants d'un utilisateur existant (ayant vu sa demande 
                de licence pour ce logiciel validé) et appuyer sur le bouton
                "\emph{Valider}".  
              & Une nouvelle interface apparaît une fois l'opération terminée
                avec un message indiquant le succès (ou non) de l'opération.
                En cas de succès, un fichier de licence doit être présent 
                dans le dossier depuis lequel l'utilisateur à éxécuté le logiciel. 
              & Exigences vérifiées & \\
            \hline
        \end{tabular} 
        \label{tab:tab3}
\end{table}
\newpage

\section{Procédure de test du CA4}
\begin{table}[!h]
        \centering
        \begin{tabular}{|m{0.6cm}|
                         >{\raggedright\arraybackslash}m{4cm}|
                         >{\raggedright\arraybackslash}m{6.4cm}|
                         >{\raggedright\arraybackslash}m{2cm}|
                         m{1cm}|}
            \hline
            \multicolumn{3}{|c|}{
                \textbf{Objet testé: } logiciel protégé par la bibliothèque ou la greffe
            } & \multicolumn{2}{|c|}{
                \textbf{Version: } version    
            } \\
            \hline
            \multicolumn{5}{|c|}{\textbf{Objectifs de test:}
                vérification du CA4 : Démarrage d'un logiciel protégé} \\
            \hline
            \multicolumn{5}{|c|}{
                \textbf{Procédure n°}4 - Test CA4 
            } \\
            \hline
            N° & Actions & Résultats attendus & Exigences & OK / NOK \\
            \hline      % start of the array
            1 & Le testeur démarre le logiciel protégé sur la machine
                cible. 
              & Un menu apparaît avec un champ de texte permettant d'écrire
                le chemin du fichier de licence, un bouton ouvrant une 
                boîte de dialogue permettant de sélectionner directement 
                le fichier de licence et un bouton avec le label "\emph{Valider}". 
              & Exigences vérifiées & \\
            \hline
            2 & Le testeur doit réaliser l'action 1 puis appuyer sur le bouton
                "\emph{Valider}"
              & En cas de succès, le logiciel protégé démarre et en cas d'échec
                l'utilisateur est ramené sur la première page avec un message 
                indiquant la raison de l'erreur. 
              & Exigences vérifiées & \\
            \hline
        \end{tabular} 
        \label{tab:tab4}
\end{table}
\newpage

\section{Procédure de test du CA5}
\begin{table}[!h]
        \centering
        \begin{tabular}{|m{0.6cm}|
                         >{\raggedright\arraybackslash}m{4cm}|
                         >{\raggedright\arraybackslash}m{6.4cm}|
                         >{\raggedright\arraybackslash}m{2cm}|
                         m{1cm}|}
            \hline
            \multicolumn{3}{|c|}{
                \textbf{Objet testé: } Application serveur
            } & \multicolumn{2}{|c|}{
                \textbf{Version: } version    
            } \\
            \hline
            \multicolumn{5}{|c|}{\textbf{Objectifs de test:}
                vérification du CA5 : Création d'un compte sur le site Web} \\
            \hline
            \multicolumn{5}{|c|}{
                \textbf{Procédure n°}5 - Test CA5
            } \\
            \hline
            N° & Actions & Résultats attendus & Exigences & OK / NOK \\
            \hline      % start of the array
            1 & Le testeur doit se rendre sur l'interface de l'application,
                dans la rubrique permettant de se créer un compte et insérer
                dans les champs "\emph{Identifiant}", "\emph{Mail}" et 
                "\emph{Mot de passe}" un identifiant, son email et un mot de 
                passe (avec un niveau de sécurité suffisant) puis appuyer sur
                le bouton "\emph{S'enregistrer}".
              & Le testeur doit constater un message lui indiquant le succès (ou non)
                de l'opération. En cas de succès, une entrée est ajoutée dans la table 
                des clients avec les champs (\emph{ClientName}, \emph{ClientMail}, 
                \emph{ClientMdp}, \emph{ClientSalt}) mis à jour. 
              & Exigences vérifiées & \\
            \hline
        \end{tabular} 
        \label{tab:tab5}
\end{table}
\newpage

\section{Procédure de test du CA6}
\begin{table}[!h]
        \centering
        \begin{tabular}{|m{0.6cm}|
                         >{\raggedright\arraybackslash}m{4cm}|
                         >{\raggedright\arraybackslash}m{6.4cm}|
                         >{\raggedright\arraybackslash}m{2cm}|
                         m{1cm}|}
            \hline
            \multicolumn{3}{|c|}{
                \textbf{Objet testé: } Application serveur
            } & \multicolumn{2}{|c|}{
                \textbf{Version: } version    
            } \\
            \hline
            \multicolumn{5}{|c|}{\textbf{Objectifs de test:} 
                vérification du CA6 : Paramétrage d'une licence} \\
            \hline
            \multicolumn{5}{|c|}{
                \textbf{Procédure n°}6 - Test CA6
            } \\
            \hline
            N° & Actions & Résultats attendus & Exigences & OK / NOK \\
            \hline      % start of the array
            1 & Le testeur doit se rendre dans le panneau de configuration
                de l'interface de l'administrateur, dans la section 
                "\emph{Paramétrage}" et choisir une licence à modifier
                dans la liste et cliquer sur le bouton "\emph{Éditer}" 
                de celui-ci. Un menu appraît alors lui permettant d'étendre
                la validité ou de modifié le nombre d'essais de celle-ci.
              & Le testeur doit alors constater un message indiquant le 
                succès (ou non) de son opération. En cas de succès la 
                table licence de la base de données est mise à jour et
                le client qui posséde la licence est notifié par mail de 
                ses changements. 
              & Exigences vérifiées & \\
            \hline
        \end{tabular} 
        \label{tab:tab6}
\end{table}
\newpage

\chapter{Jeux de données de test}

\chapter{Couverture de test}

\begin{table}[!h]
    \centering
    \begin{tabular}{|l|l|l|l|}
        \hline
            \textbf{Exigences STB} & \textbf{Méthode de vérification} & \textbf{Procédures utilisées} & \textbf{Commentaire} \\
        \hline
    \end{tabular}
    \label{tab:tab7}
\end{table}

