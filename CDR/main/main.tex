\chapter{Introduction}
Ce document présente les tests qui seront mis en place pour vérifier le bon fonctionnement et le
respect des exigences fonctionnelles et techniques de l'outil.
Ce document précise :
\begin{itemize}
    \item les conditions à satisfaire préalablement à l’exécution des tests,
    \item les moyens matériels requis (plate-forme de tests),
    \item la logique de leur déroulement (étapes successives),
    \item les conditions d’arrêt.
\end{itemize}
\medskip

Les fonctionnalités principales du logiciel (cas d'utilisations) sont les suivantes :
\begin{itemize}
    \item CA1 : Demande d'une licence,
    \item CA2 : Accord d'une licence,
    \item CA3 : Activation d'une licence,
    \item CA4 : Démarrage d'un logiciel protégé,
    \item CA5 : Création d'un compte sur le site Web,
    \item CA6 : Paramétrage d'une licence.
\end{itemize}
\medskip

Les différents modules à tester sont les suivants :
\begin{itemize}
    \item La plateforme Web (partie front-end et back-end),
    \item Le logiciel d'activation,
    \item Le système de vérification de licence.
\end{itemize}
\medskip

La partie interface et toute la logique lié à la plateforme sera bien évidemment à tester mais le
principal des tests seront sûr la sécurité de cette plateforme et de notre système de vérification
des licences.


\chapter{Documents de référence}
\textbf{Documents de spécification :}
\begin{itemize}
    \item Spécification technique des besoins
    \item Fiches techniques :
    \begin{itemize}
        \item Injection de code dans un PE
        \item Génération de licence
        \item Signature
        \item Obfuscation de code
        \item Sécurisation des bases de données
    \end{itemize}
    \item Document d'architecture technique
    \item Analyse des risques
\end{itemize}

\chapter{Terminologie}

\chapter{Environnement de test}

\section{Application Web}
La plateforme cible pour l'application web sera une machine virtuelle sur un serveur (Debian)
hébergé par l'université.

Pour être sûr de la validité de nos tests, tous les tests qui le pourront seront effectués
directement sur cette machine, ou un clone de la machine pour pouvoir revenir facilement à son
état initial, pour, par exemple, un test de déploiement de l'application.

L'API Rest de l'application web sera testé depuis nos machines depuis un système Linux et surtout
depuis un Windows car ce sera la plateforme cible pour notre outil.

\section{Logiciel d'activation}
Les tests concernant le logiciel d'activation comprendront les tests de l'API Rest et également
des tests par rapport à l'interface avec l'utilisateur. Le logiciel d'activation aura pour but
d'être utilisé sur Windows et le développement de son interface se fera à l'aide d'un outil Windows
(probablement en C\# avec Visual Studio).

Tous les tests concernant le logiciel d'activation se
feront donc sur Windows, de préférence sur plusieurs versions. Windows 10 et 11 pourront être
testés facilement, et selon si l'on dispose des machines necéssaires, Windows 7 et 8 pourront être
testés.

\section{Système de vérification}
Le système de vérification de licence devra être testé sur plusieurs systèmes. En effet, la
génération d'une licence sera effectué côté serveur, donc sur un système Linux, mais la
vérification, elle, sur un système Windows.

Il faudra donc testés les 2 cas, encore une fois sur plusieurs machines/versions.

 
\chapter{Responsabilités}
Les reponsibilités des tests seront d'abord répartis selon la répartiton des tâches, chacun
devra donc écrire des tests pour les modules qu'il dévelloppe et se créer ses propres données
de test. Puis lors de la phase d'intégration, un journal de test sera régulierement mis à jour
afin de répertorier les bugs et les régressions de la solution et ainsi identifier plus      
facilement la source de l'erreur et en trouver l'auteur afin que celui-ci la résolve.


\chapter{Stratégie de tests}

\section{Démarche général et outils} 
Nous avons choisit d'utiliser MantisBT (Mantis Bug tracker) qui est un logiciel gratuit
et open source afin de répertorier les bugs, celui-ci permet d'enregistrer la déclaration
 d'un bug, et d'y affecter un quelqu'un. Ce dernier pourra alors rendre compte de
l'avancement de sa résolution, jusqu'à sa clôture. Le déclarant de l'anomalie peut  
s'informer à tout moment via le serveur Web de l'avancement du traitement de 
son problème.  

\section{Campagne de test}
Les tests seront lancer après chaque modification sur un module ou plusieurs modules, sur
le serveur de test.

\section{Tests de l'application web}

\subsection{Partie Frontend}

Afin de tester que l'interface graphique (Vue du model MVC) fonctionne correctement, nous allons mettre en place les procédures suitantes:
\begin{itemize}

    \item vérifier la validité syntaxique des pages statiques en HTML / css via 
          le validateur du W3C. Celui ci peut être utilisé via l'interface web ou  
          être appelé directement via la commande curl:
          \begin{minted}{bash}
curl -H "Content-Type: text/html; charset=utf-8" \
    --data-binary @FILE.html \
    https://validator.w3.org/nu/?out=gnu
          \end{minted}
              
    \item vérification manuelle du rendu visuel de l'interface.
    \item tests sur différents navigateur (Chrome, firefox, opera)
            
\end{itemize}
Attention au attaques XSS

\subsection{Partie Backend}

Afin de tester que le model et le controller du model MVC fonctionne correctement, 
nous allons mettre en place les procédures suitantes:
\begin{itemize}
    
    \item réalisation de tests unitaire sur le model et le controller via 
          un script de test écrit en PHP, celui-ci pourra être lancer après 
          chaque mise à jour. Il utilisera les informations contenue dans la  
          base de données de test.
        
    \item test de l'API rest du logiciel d'activation via un script bash 
          utilisant la commande curl avec le jeu de données de la base de test.
        
    \item test du programme de génération de licence via un script bash qui
          utiilisera les données de test.
\end{itemize}
Attention au injections SQL


\chapter{Gestion des anomalies}

\chapter{Procédure de test}

\chapter{Jeux de données de test}

\chapter{Couverture de test}

